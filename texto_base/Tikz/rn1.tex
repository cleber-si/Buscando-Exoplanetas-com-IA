\begin{figure}[H]
    \centering
    \caption{Exemplo de rede neural.}
    \label{fig:rede1}
    \begin{tikzpicture}
        % Input layer neurons'number
        \newcommand{\inputnum}{3} 
        % Hidden layer 1 neurons'number
        \newcommand{\hiddennumUm}{3}
        % Hidden layer 1 neurons'number
        \newcommand{\hiddennumDois}{3}
        % Output layer neurons'number
        \newcommand{\outputnum}{2} 
        
        % Input Layer
        \foreach \i in {1,...,\inputnum}
        {
        	\node[circle, ultra thick, draw,
        		minimum size = 6mm,
        		color=blue,
        		fill=black!10] (Input-\i) at (0,-\i*1.5) {\textcolor{black}{$a_{\i}^{(1)}$}};
        }
        
        % Viés
        \node[circle, ultra thick, draw,
        		minimum size = 6mm,
        		color=red,
        		fill=black!10,
        		yshift=(\hiddennumUm-\inputnum)*5 mm
        	] (Input-0) at (0,0) {\textcolor{black}{$a_{0}^{(1)}$}};
        
        % Hidden Layer 1
        \foreach \i in {1,...,\hiddennumUm}
        {
        	\node[circle, ultra thick, draw,
        		minimum size = 6mm,
        		color=black,
        		fill=black!10,
        		yshift=(\hiddennumUm-\inputnum)*5 mm
        	] (Hidden-\i) at (2.5,-\i*1.5) {$a_{\i}^{(2)}$};
        }
        
        % Viés
        \node[circle, ultra thick, draw,
        		minimum size = 6mm,
        		color=red,
        		fill=black!10,
        		yshift=(\hiddennumUm-\inputnum)*5 mm
        	] (Hidden-0) at (2.5,0) {\textcolor{black}{$a_{0}^{(2)}$}};
        	
        % Hidden Layer 2
        \foreach \i in {1,...,\hiddennumDois}
        {
        	\node[circle, ultra thick, draw,
        		minimum size = 6mm,
        		color=black,
        		fill=black!10,
        		yshift=(\hiddennumUm-\inputnum)*5 mm
        	] (Hidden2-\i) at (5,-\i*1.5) {$a_{\i}^{(3)}$};
        }
        
        % Viés
        \node[circle, ultra thick, draw,
        		minimum size = 6mm,
        		color=red,
        		fill=black!10,
        		yshift=(\hiddennumUm-\inputnum)*5 mm
        	] (Hidden2-0) at (5,0) {\textcolor{black}{$a_{0}^{(3)}$}};
        
        % Output Layer
        \foreach \i in {1,...,\outputnum}
        {
        	\node[circle, ultra thick, draw, 
        		minimum size = 6mm,
        		color=orange,
        		fill=black!10,
        		yshift=(\outputnum-\inputnum)*5 mm
        	] (Output-\i) at (7.5,-\i*1.7) {\textcolor{black}{$a_{\i}^{(4)}$}};
        }
        
        % Connect neurons In-Hidden
        \foreach \i in {0,...,\inputnum}
        {
        	\foreach \j in {1,...,\hiddennumUm}
        	{
        		\draw[-latex, shorten >=1pt] (Input-\i) -- (Hidden-\j);	
        	}
        }
        
        % Connect neurons Hidden-Hidden2
        \foreach \i in {0,...,\hiddennumUm}
        {
        	\foreach \j in {1,...,\hiddennumUm}
        	{
        		\draw[-latex, shorten >=1pt] (Hidden-\i) -- (Hidden2-\j);
        	}
        }
        
        % Connect neurons Hidden2-Out
        \foreach \i in {0,...,\hiddennumUm}
        {
        	\foreach \j in {1,...,\outputnum}
        	{
        		\draw[-latex, shorten >=1pt] (Hidden2-\i) -- (Output-\j);
        	}
        }
        
        % Outputs
        \foreach \i in {1,...,\outputnum}
        {            
        	\draw[-latex, shorten >=1pt] (Output-\i) -- ++(1.5,0)
        		node[right]{\textcolor{black}{$\left( h_{\bm{\Theta}}(\bm{x}) \right)_{\i}$}};
        }
        
        \node[scale=0.9] at (0,-5.5) {Camada 1};
        \node[scale=0.9] at (2.5,-5.5) {Camada 2};
        \node[scale=0.9] at (5,-5.5) {Camada 3};
        \node[scale=0.9] at (7.5,-5.5) {Camada 4};
    \end{tikzpicture}
    \caption*{\small Fonte: Elaboração própria.}
\end{figure}